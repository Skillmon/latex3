% \iffalse meta-comment
%
%% File: l3text-case.dtx
%
% Copyright (C) 2020-2024 The LaTeX Project
%
% It may be distributed and/or modified under the conditions of the
% LaTeX Project Public License (LPPL), either version 1.3c of this
% license or (at your option) any later version.  The latest version
% of this license is in the file
%
%    https://www.latex-project.org/lppl.txt
%
% This file is part of the "l3kernel bundle" (The Work in LPPL)
% and all files in that bundle must be distributed together.
%
% -----------------------------------------------------------------------
%
% The development version of the bundle can be found at
%
%    https://github.com/latex3/latex3
%
% for those people who are interested.
%
%<*driver>
\documentclass[full,kernel]{l3doc}
\begin{document}
  \DocInput{\jobname.dtx}
\end{document}
%</driver>
% \fi
%
% \title{^^A
%   The \pkg{l3text-case} module\\ Text processing (case changing)^^A
% }
%
% \author{^^A
%  The \LaTeX{} Project\thanks
%    {^^A
%      E-mail:
%        \href{mailto:latex-team@latex-project.org}
%          {latex-team@latex-project.org}^^A
%    }^^A
% }
%
% \date{Released 2024-01-04}
%
% \maketitle
%
% \begin{documentation}
%
% \end{documentation}
%
% \begin{implementation}
%
% \section{\pkg{l3text-case} implementation}
%
%    \begin{macrocode}
%<*package>
%    \end{macrocode}
%
%    \begin{macrocode}
%<@@=text>
%    \end{macrocode}
%
% \subsection{Case changing}
%
% \begin{variable}{\l_text_titlecase_check_letter_bool}
%   Needed to determine the route used in titlecasing.
%    \begin{macrocode}
\bool_new:N \l_text_titlecase_check_letter_bool
\bool_set_true:N \l_text_titlecase_check_letter_bool
%    \end{macrocode}
% \end{variable}
%
% \begin{macro}[EXP]
%   {
%     \text_lowercase:n,
%     \text_uppercase:n,
%     \text_titlecase_all:n,
%     \text_titlecase_first:n
%   }
% \begin{macro}[EXP]
%   {
%     \text_lowercase:nn,
%     \text_uppercase:nn,
%     \text_titlecase_all:nn,
%     \text_titlecase_first:nn
%   }
% \begin{macro}[EXP]{\@@_change_case:nnn}
%   The user level functions here are all wrappers around the internal
%   functions for case changing.
%    \begin{macrocode}
\cs_new:Npn \text_lowercase:n #1
  { \@@_change_case:nnn { lower } { } {#1} }
\cs_new:Npn \text_uppercase:n #1
  { \@@_change_case:nnn { upper } { } {#1} }
\cs_new:Npn \text_titlecase_all:n #1
  { \@@_change_case:nnn { title } { } {#1} }
\cs_new:Npn \text_titlecase_first:n #1
  { \@@_change_case:nnnn { title } { break } { } {#1} }
\cs_new:Npn \text_lowercase:nn #1#2
  { \@@_change_case:nnn { lower } {#1} {#2} }
\cs_new:Npn \text_uppercase:nn #1#2
  { \@@_change_case:nnn { upper } {#1} {#2} }
\cs_new:Npn \text_titlecase_all:nn #1#2
  { \@@_change_case:nnn { title } {#1} {#2} }
\cs_new:Npn \text_titlecase_first:nn #1#2
  { \@@_change_case:nnnn { title } { break } {#1} {#2} }
\cs_new:Npn \@@_change_case:nnn #1#2#3
  { \@@_change_case:nnnn {#1} {#1} {#2} {#3} }
%    \end{macrocode}
% \end{macro}
% \end{macro}
% \end{macro}
%
% \begin{macro}[EXP]
%   {
%     \@@_change_case:nnnn       ,
%     \@@_change_case_auxi:nnnn  ,
%     \@@_change_case_auxii:nnnn
%   }
% \begin{macro}[EXP]{\@@_change_case_BCP:nnnn}
% \begin{macro}[EXP]{\@@_change_case_BCP:nnnw}
% \begin{macro}[EXP]{\@@_change_case_BCP:nnnnnw}
% \begin{macro}[EXP]
%   {
%     \@@_change_case_store:n, \@@_change_case_store:o,
%     \@@_change_case_store:V, \@@_change_case_store:v,
%     \@@_change_case_store:e
%   }
% \begin{macro}[EXP]{\@@_change_case_store:nw}
% \begin{macro}[EXP]{\@@_change_case_result:n} 
% \begin{macro}[EXP]{\@@_change_case_end:w}
% \begin{macro}[EXP]{\@@_change_case_loop:nnnw}
% \begin{macro}[EXP]{\@@_change_case_break:w, \@@_change_case_break_aux:w}
% \begin{macro}[EXP]
%   {
%     \@@_change_case_group_lower:nnnn ,
%     \@@_change_case_group_upper:nnnn ,
%     \@@_change_case_group_title:nnnn
%   }
% \begin{macro}[EXP]
%   {\@@_change_case_space:nnnw, \@@_change_case_space_break:nnnw}
% \begin{macro}[EXP]
%   {\@@_change_case_N_type:nnnN, \@@_change_case_N_type_aux:nnnN}
% \begin{macro}[EXP]{\@@_change_case_N_type:nnnnN}
% \begin{macro}[EXP]{\@@_change_case_math_search:nnnNNN}
% \begin{macro}[EXP]{\@@_change_case_math_loop:nnnNw}
% \begin{macro}[EXP]{\@@_change_case_math_N_type:nnnNN}
% \begin{macro}[EXP]{\@@_change_case_math_group:nnnNn}
% \begin{macro}[EXP]{\@@_change_case_math_space:nnnNw}
% \begin{macro}[EXP]{\@@_change_case_cs_check:nnnN}
% \begin{macro}[EXP]{\@@_change_case_exclude:nnnN}
% \begin{macro}[EXP]{\@@_change_case_exclude:nnnnN}
% \begin{macro}[EXP]{\@@_change_case_exclude:nnnNN}
% \begin{macro}[EXP]{\@@_change_case_exclude:nnnNw}
% \begin{macro}[EXP]{\@@_change_case_exclude:nnnNnn}
% \begin{macro}[EXP]{\@@_change_case_replace:nnnN}
% \begin{macro}[EXP]{\@@_change_case_replace:nnnn, \@@_change_case_replace:vnnn}
% \begin{macro}[EXP]{\@@_change_case_switch:nnnN}
% \begin{macro}[EXP]
%   {
%     \@@_change_case_switch_lower:nnnNnnnn ,
%     \@@_change_case_switch_upper:nnnNnnnn ,
%     \@@_change_case_switch_title:nnnNnnnn
%   }
% \begin{macro}[EXP]{\@@_change_case_skip:nnw}
% \begin{macro}[EXP]{\@@_change_case_skip_N_type:nnN}
% \begin{macro}[EXP]{\@@_change_case_skip_group:nnn}
% \begin{macro}[EXP]{\@@_change_case_skip_space:nnw}
% \begin{macro}[EXP]
%   {
%     \@@_change_case_letterlike_lower:nnnN ,
%     \@@_change_case_letterlike_upper:nnnN ,
%     \@@_change_case_letterlike_title:nnnN
%   }
% \begin{macro}[EXP]{\@@_change_case_letterlike:nnnnnN}
% \begin{macro}[EXP]
%   {
%     \@@_change_case_custom_lower:nnnn ,
%     \@@_change_case_custom_title:nnnn ,
%     \@@_change_case_custom_upper:nnnn
%   }
% \begin{macro}[EXP]{\@@_change_case_custom:nnnnn}
% \begin{macro}[EXP]
%   {
%     \@@_change_case_codepoint_lower:nnnn ,
%     \@@_change_case_codepoint_upper:nnnn ,
%     \@@_change_case_codepoint_title:nnnn
%   }
% \begin{macro}[EXP]{\@@_change_case_lower_sigma:nnnnn}
% \begin{macro}[EXP]{\@@_change_case_lower_sigma:nnnnw}
% \begin{macro}[EXP]{\@@_change_case_lower_sigma:nnnnN}
% \begin{macro}[EXP]
%   {
%     \@@_change_case_codepoint_title_auxi:nnnn ,
%     \@@_change_case_codepoint_title_auxii:nnnn
%   }
% \begin{macro}[EXP]{\@@_change_case_codepoint_title:nnn}
% \begin{macro}[EXP]{\@@_change_case_codepoint:nnnnn}
% \begin{macro}[EXP]{\@@_change_case_codepoint:nn}
% \begin{macro}[EXP]
%   {
%     \@@_change_case_codepoint:nnn ,
%     \@@_change_case_codepoint:fnn ,
%     \@@_change_case_codepoint_aux:nnn
%   }
% \begin{macro}[EXP]{\@@_change_case_codepoint_aux:nnn}
% \begin{macro}[EXP]{\@@_change_case_codepoint_aux:nn}
% \begin{macro}[EXP]{\@@_change_case_catcode:nn}
% \begin{macro}[EXP]
%   {
%     \@@_change_case_next_lower:nnn ,
%     \@@_change_case_next_upper:nnn ,
%     \@@_change_case_next_title:nnn ,
%     \@@_change_case_next_end:nnn
%   }
%   As for the expansion code, the business end of case changing is the
%   handling of \texttt{N}-type tokens. First, we expand the input fully
%   (so the loops here don't need to worry about awkward look-aheads and the
%   like). Then we split into the different paths.
%
%   The code here needs to be \texttt{f}-type expandable to deal with the
%   situation where case changing is applied in running text. There, we
%   might have case changing as a document command and the text containing
%   other non-expandable document commands.
%   \begin{verbatim}
%     \cs_set_eq:NN \MakeLowercase \text_lowercase
%     ...
%     \MakeLowercase{\enquote*{A} text}
%   \end{verbatim}
%   If we use an \texttt{e}-type expansion and wrap each token in
%   \cs{exp_not:n}, that would explode: the document command grabs
%   \cs{exp_not:n} as an argument, and things go badly wrong. So we have to
%   wrap the entire result in exactly one \cs{exp_not:n}, or rather in the
%   kernel version.
%    \begin{macrocode}
\cs_new:Npn \@@_change_case:nnnn #1#2#3#4
  {
    \__kernel_exp_not:w \exp_after:wN
      {
        \exp:w
        \exp_args:Ne \@@_change_case_auxi:nnnn
          { \text_expand:n {#4} }
          {#1} {#2} {#3}
      }
  }
\cs_new:Npn \@@_change_case_auxi:nnnn #1#2#3#4
  {
    \exp_args:No \@@_change_case_BCP:nnnn
      { \tl_to_str:n {#4} } {#1} {#2} {#3}
  }
\cs_new:Npe \@@_change_case_BCP:nnnn #1#2#3#4
  {
    \exp_not:N \@@_change_case_BCP:nnnw
      {#2} {#3} {#4} #1 \tl_to_str:n { -x- -x- } \exp_not:N \q_@@_stop
  }
\use:e
  {
    \cs_new:Npn \exp_not:N \@@_change_case_BCP:nnnw
      #1#2#3#4 \tl_to_str:n { -x- } #5 \tl_to_str:n { -x- } #6
      \exp_not:N \q_@@_stop
  }
  { \@@_change_case_BCP:nnnnnw {#1} {#2} {#3} {#5} {#4} #4 - \q_@@_stop }
\cs_new:Npn \@@_change_case_BCP:nnnnnw #1#2#3#4#5#6 - #7 \q_@@_stop
  {
    \bool_lazy_or:nnTF
      { \cs_if_exist_p:c { @@_change_case_ #2 _ #6 -x- #4 :nnnnn } }
      { \tl_if_exist_p:c { l_@@_ #2 case_special_ #6 -x- #4 _tl } }
      { \@@_change_case_auxii:nnnn {#1} {#2} {#3} { #6 -x- #4 } }
      {
        \cs_if_exist:cTF { @@_change_case_ #2 _ #6 :nnnnn }
          { \@@_change_case_auxii:nnnn {#1} {#2} {#3} {#6} }
          { \@@_change_case_auxii:nnnn {#1} {#2} {#3} {#5} }
      }
  }
\cs_new:Npn \@@_change_case_auxii:nnnn #1#2#3#4
  {
    \group_align_safe_begin:
    \cs_if_exist_use:c { @@_change_case_boundary_ #2 _ #4 :Nnnnw }
    \@@_change_case_loop:nnnw {#2} {#3} {#4} #1
      \q_@@_recursion_tail \q_@@_recursion_stop
    \@@_change_case_result:n { }
  }
%    \end{macrocode}
%   As for expansion, collect up the tokens for future use.
%    \begin{macrocode}
\cs_new:Npn \@@_change_case_store:n #1
  { \@@_change_case_store:nw {#1} }
\cs_generate_variant:Nn \@@_change_case_store:n { o , e , V , v }
\cs_new:Npn \@@_change_case_store:nw #1#2 \@@_change_case_result:n #3
  { #2 \@@_change_case_result:n { #3 #1 } }
\cs_new:Npn \@@_change_case_end:w #1 \@@_change_case_result:n #2
  {
    \group_align_safe_end:
    \exp_end:
    #2
  }
%    \end{macrocode}
%   The main loop is the standard \texttt{tl action} type.
%    \begin{macrocode}
\cs_new:Npn \@@_change_case_loop:nnnw #1#2#3#4 \q_@@_recursion_stop
  {
    \tl_if_head_is_N_type:nTF {#4}
      { \@@_change_case_N_type:nnnN }
      {
        \tl_if_head_is_group:nTF {#4}
          { \use:c { @@_change_case_group_ #1 :nnnn } }
          { \@@_change_case_space:nnnw }
      }
    {#1} {#2} {#3} #4 \q_@@_recursion_stop
  }
\cs_new:Npn \@@_change_case_break:w
  { \@@_change_case_break_aux:w \prg_do_nothing: }
\cs_new:Npn \@@_change_case_break_aux:w
  #1 \q_@@_recursion_tail \q_@@_recursion_stop
  {
    \@@_change_case_store:o {#1}
    \@@_change_case_end:w
  }
%    \end{macrocode}
%   For a group, we \emph{could} worry about whether this contains a character
%   or not. However, that would make life very complex for little gain: exactly
%   what a first character is is rather weakly-defined anyway. So if there is
%   a group, we simply assume that a character has been seen, and for title
%   case we switch to the \enquote{rest of the tokens} situation. To avoid
%   having too much testing, we use a two-step process here to allow the
%   titlecase functions to be separate.
%    \begin{macrocode}
\cs_new:Npn \@@_change_case_group_lower:nnnn #1#2#3#4
  {
    \@@_change_case_store:o
      {
        \exp_after:wN
          {
            \exp:w
            \@@_change_case_auxii:nnnn {#4} {#1} {#2} {#3}
          }
      }
    \@@_change_case_loop:nnnw {#1} {#2} {#3}
  }
\cs_new_eq:NN \@@_change_case_group_upper:nnnn
  \@@_change_case_group_lower:nnnn
\cs_new:Npn \@@_change_case_group_title:nnnn #1#2#3#4
  {
    \@@_change_case_store:o
      {
        \exp_after:wN
          {
            \exp:w
            \@@_change_case_auxii:nnnn {#4} {#1} {#2} {#3}
          }
      }
    \@@_change_case_skip:nnw {#2} {#3}
  }
\use:e
  {
    \cs_new:Npn \exp_not:N \@@_change_case_space:nnnw #1#2#3 \c_space_tl
  }
  {
    \@@_change_case_store:n { ~ }
    \cs_if_exist_use:cF { @@_change_case_space_ #2 :nnn }
      {
        \cs_if_exist_use:c { @@_change_case_boundary_ #1 _ #3 :Nnnnw }
        \@@_change_case_loop:nnnw
      }
        {#2} {#2} {#3}
  }
\cs_new:Npn \@@_change_case_space_break:nnn #1#2#3
  { \@@_change_case_break:w }
%    \end{macrocode}
%   The first step of handling \texttt{N}-type tokens is to filter out the
%   end-of-loop. That has to be done separately from the first real step
%   as otherwise we pick up the wrong delimiter. The loop here is the same
%   as the \texttt{expand} one, just passing the additional data long. If no
%   close-math token is found then the final clean-up is forced
%   (i.e.~there is no assumption of \enquote{well-behaved} input in terms of
%   math mode).
%    \begin{macrocode}
\cs_new:Npn \@@_change_case_N_type:nnnN #1#2#3#4
  {
    \@@_if_q_recursion_tail_stop_do:Nn #4
      { \@@_change_case_end:w }
    \@@_change_case_N_type_aux:nnnN {#1} {#2} {#3} #4
  }
\cs_new:Npn \@@_change_case_N_type_aux:nnnN #1#2#3#4
  {
    \exp_args:NV \@@_change_case_N_type:nnnnN
      \l_text_math_delims_tl {#1} {#2} {#3} #4
  }
\cs_new:Npn \@@_change_case_N_type:nnnnN #1#2#3#4#5
  {
    \@@_change_case_math_search:nnnNNN {#2} {#3} {#4} #5 #1
      \q_@@_recursion_tail \q_@@_recursion_tail
      \q_@@_recursion_stop
  }
\cs_new:Npn \@@_change_case_math_search:nnnNNN #1#2#3#4#5#6
  {
    \@@_if_q_recursion_tail_stop_do:Nn #5
      { \@@_change_case_cs_check:nnnN {#1} {#2} {#3} #4 }
    \token_if_eq_meaning:NNTF #4 #5
      {
        \@@_use_i_delimit_by_q_recursion_stop:nw
          {
            \@@_change_case_store:n {#4}
            \@@_change_case_math_loop:nnnNw {#1} {#2} {#3} #6
          }
      }
      { \@@_change_case_math_search:nnnNNN {#1} {#2} {#3} #4 }
  }
\cs_new:Npn \@@_change_case_math_loop:nnnNw #1#2#3#4#5 \q_@@_recursion_stop
  {
    \tl_if_head_is_N_type:nTF {#5}
      { \@@_change_case_math_N_type:nnnNN }
      {
        \tl_if_head_is_group:nTF {#5}
          { \@@_change_case_math_group:nnnNn }
          { \@@_change_case_math_space:nnnNw }
      }
    {#1} {#2} {#3} #4 #5 \q_@@_recursion_stop
  }
\cs_new:Npn \@@_change_case_math_N_type:nnnNN #1#2#3#4#5
  {
    \@@_if_q_recursion_tail_stop_do:Nn #5
      { \@@_change_case_end:w }
    \@@_change_case_store:n {#5}
    \token_if_eq_meaning:NNTF #5 #4
      { \@@_change_case_loop:nnnw {#1} {#2} {#3} }
      { \@@_change_case_math_loop:nnnNw {#1} {#2} {#3} #4 }
  }
\cs_new:Npn \@@_change_case_math_group:nnnNn #1#2#3#4#5
  {
    \@@_change_case_store:n { {#5} }
    \@@_change_case_math_loop:nnnNw {#1} {#2} {#3} #4
  }
\use:e
  {
    \cs_new:Npn \exp_not:N \@@_change_case_math_space:nnnNw #1#2#3#4
      \c_space_tl
  }
  {
    \@@_change_case_store:n { ~ }
    \@@_change_case_math_loop:nnnNw {#1} {#2} {#3} #4
  }
%    \end{macrocode}
%   Once potential math-mode cases are filtered out the next stage is to
%   test if the token grabbed is a control sequence: the two routes the code
%   may take are then very different.
%    \begin{macrocode}
\cs_new:Npn \@@_change_case_cs_check:nnnN #1#2#3#4
  {
    \token_if_cs:NTF #4
      { \@@_change_case_exclude:nnnN {#1} {#2} {#3} }
      {
        \@@_codepoint_process:nN
          { \use:c { @@_change_case_custom_ #1 :nnnn } {#1} {#2} {#3} }
      }
        #4
  }
%    \end{macrocode}
%   To deal with a control sequence there is first a need to test if it is
%   on the list which indicate that case changing should be skipped. That's
%   done using a loop as for the other special cases. If a hit is found then
%   the argument is grabbed and passed through as-is.
%    \begin{macrocode}
\cs_new:Npn \@@_change_case_exclude:nnnN #1#2#3#4
  {
    \exp_args:Ne \@@_change_case_exclude:nnnnN
      {
        \exp_not:V \l_text_math_arg_tl
        \exp_not:V \l_text_case_exclude_arg_tl
      }
      {#1} {#2} {#3} #4
  }
\cs_new:Npn \@@_change_case_exclude:nnnnN #1#2#3#4#5
  {
    \@@_change_case_exclude:nnnNN {#2} {#3} {#4} #5 #1 
      \q_@@_recursion_tail \q_@@_recursion_stop
  }
\cs_new:Npn \@@_change_case_exclude:nnnNN #1#2#3#4#5
  {
    \@@_if_q_recursion_tail_stop_do:Nn #5
      { \@@_change_case_replace:nnnN {#1} {#2} {#3} #4 }
    \str_if_eq:nnTF {#4} {#5}
      {
        \@@_use_i_delimit_by_q_recursion_stop:nw
          { \@@_change_case_exclude:nnnNw {#1} {#2} {#3} #4 }
      }
      { \@@_change_case_exclude:nnnNN {#1} {#2} {#3} #4 }
  }
\cs_new:Npn \@@_change_case_exclude:nnnNw #1#2#3#4#5#
  { \@@_change_case_exclude:nnnNnn {#1} {#2} {#3} {#4} {#5} }
\cs_new:Npn \@@_change_case_exclude:nnnNnn #1#2#3#4#5#6
  {
    \tl_if_blank:nTF {#5}
      { \@@_change_case_store:n { #4 {#6} } }
      {
        \@@_change_case_store:o
          {
            \exp_after:wN #4
              \exp:w \@@_change_case_auxii:nnnn {#5} {#1} {#2} {#3}
              {#6}
          }
      }
    \@@_change_case_loop:nnnw {#1} {#2} {#3}
  }
%    \end{macrocode}
%   Deal with any specialist replacement for case changing.
%    \begin{macrocode}
\cs_new:Npn \@@_change_case_replace:nnnN #1#2#3#4
  {
    \cs_if_exist:cTF { l_@@_case_ \token_to_str:N #4 _tl }
      {
        \@@_change_case_replace:vnnn
          { l_@@_case_ \token_to_str:N #4 _tl } {#1} {#2} {#3}
      }
      { \@@_change_case_switch:nnnN {#1} {#2} {#3} #4 }
  }
\cs_new:Npn \@@_change_case_replace:nnnn #1#2#3#4
  { \@@_change_case_loop:nnnw {#2} {#3} {#4} #1 }
\cs_generate_variant:Nn \@@_change_case_replace:nnnn { v }
%    \end{macrocode}
%   Allow for manually-controlled case switching.
%    \begin{macrocode}
\cs_new:Npn \@@_change_case_switch:nnnN #1#2#3#4
  {
    \cs_if_eq:NNTF #4 \text_case_switch:nnnn
      { \use:c { @@_change_case_switch_ #1 :nnnNnnnn  } }
      { \use:c { @@_change_case_letterlike_ #1 :nnnN } }
        {#1} {#2} {#3} #4
  }
\cs_new:Npn \@@_change_case_switch_lower:nnnNnnnn #1#2#3#4#5#6#7#8
  {
    \@@_change_case_store:n {#7}
    \@@_change_case_loop:nnnw {#1} {#2} {#3}
  }
\cs_new:Npn \@@_change_case_switch_upper:nnnNnnnn #1#2#3#4#5#6#7#8
  {
    \@@_change_case_store:n {#6}
    \@@_change_case_loop:nnnw {#1} {#2} {#3}
  }
\cs_new:Npn \@@_change_case_switch_title:nnnNnnnn #1#2#3#4#5#6#7#8
  {
    \@@_change_case_store:n {#8}
    \@@_change_case_skip:nnw {#2} {#3}
  }
%    \end{macrocode}
%   Skip over material quickly after titlecase-first-only initials
%    \begin{macrocode}
\cs_new:Npn \@@_change_case_skip:nnw #1#2#3 \q_@@_recursion_stop
  {
    \tl_if_head_is_N_type:nTF {#3}
      { \@@_change_case_skip_N_type:nnN }
      {
        \tl_if_head_is_group:nTF {#3}
          { \@@_change_case_skip_group:nnn }
          { \@@_change_case_skip_space:nnw }
      }
        {#1} {#2} #3 \q_@@_recursion_stop
  }
\cs_new:Npn \@@_change_case_skip_N_type:nnN #1#2#3
  {
    \@@_if_q_recursion_tail_stop_do:Nn #3
      { \@@_change_case_end:w }
    \@@_change_case_store:n {#3}
    \@@_change_case_skip:nnw {#1} {#2}
  }
\cs_new:Npn \@@_change_case_skip_group:nnn #1#2#3
  {
    \@@_change_case_store:n { {#3} }
    \@@_change_case_skip:nnw {#1} {#2}
  }
\cs_new:Npn \@@_change_case_skip_space:nnw #1#2
  { \@@_change_case_space:nnnw {#1} {#1} {#2} }
%    \end{macrocode}
%  Letter-like commands may still be present: they are set up using a simple
%  lookup approach, so can easily be handled with no loop. If there is no
%  hit, we are at the end of the process: we loop around. Letter-like chars
%  are all available only in upper- and lowercase, so titlecasing maps to the
%  uppercase version.
%    \begin{macrocode}
\cs_new:Npn \@@_change_case_letterlike_lower:nnnN #1#2#3#4
  { \@@_change_case_letterlike:nnnnnN {#1} {#1} {#1} {#2} {#3} #4 }
\cs_new_eq:NN \@@_change_case_letterlike_upper:nnnN
  \@@_change_case_letterlike_lower:nnnN
\cs_new:Npn \@@_change_case_letterlike_title:nnnN #1#2#3#4
  { \@@_change_case_letterlike:nnnnnN { upper } { end } {#1} {#2} {#3} #4 }
\cs_new:Npn \@@_change_case_letterlike:nnnnnN #1#2#3#4#5#6
  {
    \cs_if_exist:cTF { c_@@_ #1 case_ \token_to_str:N #6 _tl }
      {
        \@@_change_case_store:v
          { c_@@_ #1 case_ \token_to_str:N #6 _tl }
        \use:c { @@_change_case_next_ #2 :nnn } {#2} {#4} {#5}
      }
      {
        \@@_change_case_store:n {#6}
        \cs_if_exist:cTF
          {
            c_@@_
            \str_if_eq:nnTF {#1} { lower } { upper } { lower }
            case_ \token_to_str:N #6 _tl
          }
          { \use:c { @@_change_case_next_ #2 :nnn } {#2} {#4} {#5} }
          { \@@_change_case_loop:nnnw {#3} {#4} {#5} }
      }
  }
%    \end{macrocode}
%  Check for a customised codepoint result.
%    \begin{macrocode}
\cs_new:Npn \@@_change_case_custom_lower:nnnn #1#2#3#4
  {
    \@@_change_case_custom:nnnnnn {#1} {#1} {#2} {#3} {#4}
      { \use:c { @@_change_case_codepoint_ #1 :nnnn } {#1} {#2} {#3} {#4} }
  }
\cs_new_eq:NN \@@_change_case_custom_upper:nnnn
  \@@_change_case_custom_lower:nnnn
\cs_new:Npn \@@_change_case_custom_title:nnnn #1#2#3#4
  {
    \@@_change_case_custom:nnnnnn { title } {#1} {#2} {#3} {#4}
      {
        \@@_change_case_custom:nnnnnn { upper } {#1} {#2} {#3} {#4}
          { \use:c { @@_change_case_codepoint_ #1 :nnnn } {#1} {#2} {#3} {#4} }
      }
  }
\cs_new:Npn \@@_change_case_custom:nnnnnn #1#2#3#4#5#6
  {
    \tl_if_exist:cTF { l_@@_ #1 case _ \tl_to_str:n {#5} _ #4 _tl }
      {
        \@@_change_case_replace:vnnn
          { l_@@_ #1 case _ \tl_to_str:n {#5} _ #4 _tl } {#2} {#3} {#4}
      }
      {
        \tl_if_exist:cTF { l_@@_ #1 case _ \tl_to_str:n {#5} _tl }
          {
            \@@_change_case_replace:vnnn
              { l_@@_ #1 case _ \tl_to_str:n {#5} _tl } {#2} {#3} {#4}
          }
          {#6}
      }
  }
%    \end{macrocode}
%   For upper- and lowercase changes, once we get to this stage there are only
%   a couple of questions remaining: is there a language-specific mapping and
%   is there the special case of a terminal sigma. If not, then we pass to
%   a simple codepoint mapping.
%    \begin{macrocode}
\cs_new:Npn \@@_change_case_codepoint_lower:nnnn #1#2#3#4
  {
    \cs_if_exist_use:cF { @@_change_case_lower_ #3 :nnnnn }
      { \@@_change_case_lower_sigma:nnnnn }
        {#1} {#1} {#2} {#3} {#4}
  }
\cs_new:Npn \@@_change_case_codepoint_upper:nnnn #1#2#3#4
  {
    \cs_if_exist_use:cF { @@_change_case_upper_ #3 :nnnnn }
      { \@@_change_case_codepoint:nnnnn }
        {#1} {#1} {#2} {#3} {#4}
  }
%    \end{macrocode}
%   If the current character is an uppercase sigma, the a check is made on the
%   next item in the input.  If it is \texttt{N}-type and not a control sequence
%   then there is a look-ahead phase: the logic here is simply based on letters
%   or actives (to cover $8$-bit engines).
%    \begin{macrocode}
\cs_new:Npn \@@_change_case_lower_sigma:nnnnn #1#2#3#4#5
  {
    \@@_codepoint_compare:nNnTF {#5} = { "03A3 }
      { \@@_change_case_lower_sigma:nnnnw {#2} }
      { \@@_change_case_codepoint:nnnnn {#1} {#2} }
        {#3} {#4} {#5}
  }
\cs_new:Npn \@@_change_case_lower_sigma:nnnnw #1#2#3#4#5 \q_@@_recursion_stop
  {
    \tl_if_head_is_N_type:nTF {#5}
      { \@@_change_case_lower_sigma:nnnnN {#4} }
      {
        \@@_change_case_store:e
          { \codepoint_generate:nn { "03C2 } { \@@_char_catcode:N #4 } }
        \@@_change_case_loop:nnnw
      }
        {#1} {#2} {#3} #5 \q_@@_recursion_stop
  }
\cs_new:Npn \@@_change_case_lower_sigma:nnnnN #1#2#3#4#5
  {
    \@@_change_case_store:e
      {
        \bool_lazy_or:nnTF
          { \token_if_letter_p:N #5 }
          {
            \bool_lazy_and_p:nn
              { \token_if_active_p:N #5 }
              { \int_compare_p:nNn {`#5} > { "80 } }
          }
          { \codepoint_generate:nn { "03C3 } { \@@_char_catcode:N #1 } }
          { \codepoint_generate:nn { "03C2 } { \@@_char_catcode:N #1 } }
      }
    \@@_change_case_loop:nnnw {#2} {#3} {#4} #5
  }
%    \end{macrocode}
%   For titlecasing, we need to obtain the general category of the current
%   codepoint.
%    \begin{macrocode}
\cs_new:Npn \@@_change_case_codepoint_title:nnnn #1#2#3#4
  {
    \bool_if:NTF \l_text_titlecase_check_letter_bool
      {
        \exp_args:Ne \@@_change_case_codepoint_title_auxi:nnnn
          {
            \codepoint_to_category:n
              { \@@_codepoint_from_chars:Nw #4 }
          }
      }
      { \@@_change_case_codepoint_title:nnn }
        {#2} {#3} {#4}
  }
\cs_new:Npn \@@_change_case_codepoint_title_auxi:nnnn #1#2#3#4
  {
    \tl_if_head_eq_charcode:nNTF {#1} { L }
      { \@@_change_case_codepoint_title:nnn }
      { \@@_change_case_codepoint_title_auxii:nnnn { title } }
        {#2} {#3} {#4}
  }
\cs_new:Npn \@@_change_case_codepoint_title:nnn #1#2#3
  { \@@_change_case_codepoint_title_auxii:nnnn { end } {#1} {#2} {#3} }
\cs_new:Npn \@@_change_case_codepoint_title_auxii:nnnn #1#2#3#4
  {
    \cs_if_exist_use:cF { @@_change_case_title_ #3 :nnnnn }
      {
        \cs_if_exist_use:cF { @@_change_case_upper_ #3 :nnnnn }
          { \@@_change_case_codepoint:nnnnn }
      }
        { title } {#1} {#2} {#3} {#4}
  }
\cs_new:Npn \@@_change_case_codepoint:nnnnn #1#2#3#4#5
  {
    \bool_lazy_and:nnTF
      { \tl_if_single_p:n {#5} }
      { \token_if_active_p:N #5 }
      { \@@_change_case_store:n {#5} }
      {
        \@@_change_case_store:e
          { \@@_change_case_codepoint:nn {#1} {#5} }
      }
    \use:c { @@_change_case_next_ #2 :nnn } {#2} {#3} {#4}
  }
\cs_new:Npn \@@_change_case_codepoint:nn #1#2
  {
    \@@_change_case_codepoint:fnn
      { \int_eval:n { \@@_codepoint_from_chars:Nw #2 } } {#1} {#2}
  }
\cs_new:Npn \@@_change_case_codepoint:nnn #1#2#3
  {
    \exp_args:Ne \@@_change_case_codepoint_aux:nn
      { \__kernel_codepoint_case:nn { #2 case } {#1} } {#3}
  }
\cs_generate_variant:Nn \@@_change_case_codepoint:nnn { f }
%    \end{macrocode}
%   Avoid high chars with p\TeX{}.
%    \begin{macrocode}
\sys_if_engine_ptex:T
  {
    \cs_new_eq:NN \@@_change_case_codepoint_aux:nnn
      \@@_change_case_codepoint:nnn
    \cs_gset:Npn \@@_change_case_codepoint:nnn #1#2#3
      {
        \int_compare:nNnTF {#1} = { -1 }
          { \exp_not:n {#3} }
          { \@@_change_case_codepoint_aux:nnn {#1} {#2} {#3} }
      }
  }
\cs_new:Npn \@@_change_case_codepoint_aux:nn #1#2
  {
    \use:e { \@@_change_case_codepoint_aux:nnnn #1 {#2} }
  }
\cs_new:Npn \@@_change_case_codepoint_aux:nnnn #1#2#3#4
  {
    \@@_codepoint_compare:nNnTF {#4} = {#1}
      { \exp_not:n {#4} }
      {
        \codepoint_generate:nn {#1}
          { \@@_change_case_catcode:nn {#4} {#1} }
        \tl_if_blank:nF {#2}
          {
            \codepoint_generate:nn {#2}
              { \char_value_catcode:n {#2} }
            \tl_if_blank:nF {#3}
              {
                \codepoint_generate:nn {#3}
                  { \char_value_catcode:n {#3} }
              }
          }
      }
  }
%    \end{macrocode}
%   We need to ensure that only valid catcode-extraction is attempted. That's
%   fine with Unicode engines but needs a bit of work with 8-bit ones. The
%   logic is that if the original codepoint was in the ASCII range, we keep
%   the catcode. Otherwise, if the target is in the ASCII range, we use
%   the standard catcode. If neither are true, we set as 13 on the grounds that
%   this will be what is used anyway!
%    \begin{macrocode}
\bool_lazy_or:nnTF
  { \sys_if_engine_luatex_p: }
  { \sys_if_engine_xetex_p: }
  {
    \cs_new:Npn \@@_change_case_catcode:nn #1#2
      { \@@_char_catcode:N #1 }
  }
  {
    \cs_new:Npn \@@_change_case_catcode:nn #1#2
      {
        \@@_codepoint_compare:nNnTF {#1} < { "80 }
          { \@@_char_catcode:N #1 }
          {
            \int_compare:nNnTF {#2} < { "80 }
              { \char_value_catcode:n {#2} }
              { 13 }
          }
      }
  }
\cs_new:Npn \@@_change_case_next_lower:nnn #1#2#3
  { \@@_change_case_loop:nnnw {#1} {#2} {#3} }
\cs_new_eq:NN \@@_change_case_next_upper:nnn
  \@@_change_case_next_lower:nnn
\cs_new_eq:NN \@@_change_case_next_title:nnn
  \@@_change_case_next_lower:nnn
\cs_new:Npn \@@_change_case_next_end:nnn #1#2#3
  { \@@_change_case_skip:nnw {#2} {#3} }
%    \end{macrocode}
% \end{macro}
% \end{macro}
% \end{macro}
% \end{macro}
% \end{macro}
% \end{macro}
% \end{macro}
% \end{macro}
% \end{macro}
% \end{macro}
% \end{macro}
% \end{macro}
% \end{macro}
% \end{macro}
% \end{macro}
% \end{macro}
% \end{macro}
% \end{macro}
% \end{macro}
% \end{macro}
% \end{macro}
% \end{macro}
% \end{macro}
% \end{macro}
% \end{macro}
% \end{macro}
% \end{macro}
% \end{macro}
% \end{macro}
% \end{macro}
% \end{macro}
% \end{macro}
% \end{macro}
% \end{macro}
% \end{macro}
% \end{macro}
% \end{macro}
% \end{macro}
% \end{macro}
% \end{macro}
% \end{macro}
% \end{macro}
% \end{macro}
% \end{macro}
% \end{macro}
% \end{macro}
% \end{macro}
% \end{macro}
% \end{macro}
% \end{macro}
%
% \begin{macro}{\text_declare_case_equivalent:Nn}
%  Create equivalents to allow replacement.
%    \begin{macrocode}
\cs_new_protected:Npn \text_declare_case_equivalent:Nn #1#2
  {
    \tl_clear_new:c { l_@@_case_ \token_to_str:N #1 _tl }
    \tl_set:cn { l_@@_case_ \token_to_str:N #1 _tl } {#2}
  }
%    \end{macrocode}
% \end{macro}
%
% \begin{macro}
%   {
%     \text_declare_lowercase_mapping:nn ,
%     \text_declare_titlecase_mapping:nn ,
%     \text_declare_uppercase_mapping:nn
%   }
% \begin{macro}
%   {\@@_declare_case_mapping:nnn, \@@_declare_case_mapping_aux:nnn}
% \begin{macro}
%   {
%     \text_declare_lowercase_mapping:nnn ,
%     \text_declare_titlecase_mapping:nnn ,
%     \text_declare_uppercase_mapping:nnn
%   }
% \begin{macro}
%   {\@@_declare_case_mapping:nnnn, \@@_declare_case_mapping_aux:nnnn}
%   Codepoint customisation.
%    \begin{macrocode}
\cs_new_protected:Npn \text_declare_lowercase_mapping:nn #1#2
  { \@@_declare_case_mapping:nnn { lower } {#1} {#2} }
\cs_new_protected:Npn \text_declare_titlecase_mapping:nn #1#2
  { \@@_declare_case_mapping:nnn { title } {#1} {#2} }
\cs_new_protected:Npn \text_declare_uppercase_mapping:nn #1#2
  { \@@_declare_case_mapping:nnn { upper } {#1} {#2} }
\cs_new_protected:Npn \@@_declare_case_mapping:nnn #1#2#3
  {
    \exp_args:Ne \@@_declare_case_mapping_aux:nnn
      { \codepoint_str_generate:n {#2} } {#1} {#3}
  }
\cs_new_protected:Npn \@@_declare_case_mapping_aux:nnn #1#2#3
  {
    \tl_clear_new:c { l_@@_ #2 case _ #1 _tl }
    \tl_set:cn { l_@@_ #2 case _ #1 _ tl } {#3}
  }
\cs_new_protected:Npn \text_declare_lowercase_mapping:nnn #1#2#3
  { \@@_declare_case_mapping:nnnn { lower } {#1} {#2} {#3} }
\cs_new_protected:Npn \text_declare_titlecase_mapping:nnn #1#2#3
  { \@@_declare_case_mapping:nnnn { title } {#1} {#2} {#3} }
\cs_new_protected:Npn \text_declare_uppercase_mapping:nnn #1#2#3
  { \@@_declare_case_mapping:nnnn { upper } {#1} {#2} {#3} }
\cs_new_protected:Npn \@@_declare_case_mapping:nnnn #1#2#3#4
  {
    \exp_args:Ne \@@_declare_case_mapping_aux:nnnn
      { \codepoint_str_generate:n {#3} } {#1} {#2} {#4}
  }
\cs_new_protected:Npn \@@_declare_case_mapping_aux:nnnn #1#2#3#4
  {
    \tl_clear_new:c { l_@@_ #2 case _ #1 _ #3 _tl }
    \tl_set:cn { l_@@_ #2 case _ #1 _ #3 _ tl } {#4}
    \tl_clear_new:c { l_@@_ #2 case_special_ #3 _tl }
  }
%    \end{macrocode}
% \end{macro}
% \end{macro}
% \end{macro}
% \end{macro}
%
% \begin{macro}{\text_case_switch:nnnn}
% \begin{macro}{\@@_case_switch_marker:}
%   Set up the mechanism for manual case switching.
%    \begin{macrocode}
\cs_new:Npn \text_case_switch:nnnn #1#2#3#4
  {
    \@@_case_switch_marker:
    #1
  }
\cs_new:Npn \@@_case_switch_marker: { }
%    \end{macrocode}
% \end{macro}
% \end{macro}
%
% \begin{macro}[EXP]{\@@_change_case_generate:n}
%   A utility.
%    \begin{macrocode}
\cs_new:Npn \@@_change_case_generate:n #1
  { \codepoint_generate:nn {#1} { \char_value_catcode:n {#1} } }
%    \end{macrocode}
% \end{macro}
%
% \begin{macro}[EXP]
%   {
%      \@@_change_case_upper_de-x-eszett:nnnnn,
%      \@@_change_case_upper_de-alt:nnnnn
%   }
%   A simple alternative version for German.
%    \begin{macrocode}
\cs_new:cpn { @@_change_case_upper_de-x-eszett:nnnnn } #1#2#3#4#5
  {
    \@@_codepoint_compare:nNnTF {#5} = { "00DF }
      {
        \@@_change_case_store:e
          {
            \codepoint_generate:nn { "1E9E }
              { \@@_change_case_catcode:nn {#5} { "1E9E } }
          }
        \use:c { @@_change_case_next_ #2 :nnn }
          {#2} {#3} {#4}
      }
      { \@@_change_case_codepoint:nnnnn {#1} {#2} {#3} {#4} {#5} }
  }
\cs_new_eq:cc { @@_change_case_upper_de-alt:nnnnn }
  { @@_change_case_upper_de-x-eszett:nnnnn }
%    \end{macrocode}
% \end{macro}
%
% \begin{macro}[EXP]
%   {
%     \@@_change_case_upper_el:nnnnn        ,
%     \@@_change_case_upper_el-x-iota:nnnnn ,
%     \@@_change_case_upper_el_aux:nnnnn
%   }
% \begin{macro}[EXP]{\@@_change_case_upper_el:nnnn}
% \begin{macro}[EXP]{\@@_change_case_upper_el:nnnnw}
% \begin{macro}[EXP]
%   {\@@_change_case_upper_el:nnnnN, \@@_change_case_upper_el_aux:nnnnN}
% \begin{macro}[EXP]{\@@_change_case_upper_el_ypogegrammeni:nnnnnnw}
% \begin{macro}[EXP]{\@@_change_case_upper_el_ypogegrammeni:nnnnnnN}
% \begin{macro}[EXP]{\@@_change_case_upper_el_ypogegrammeni:nnnnnnn}
% \begin{macro}[EXP]{\@@_change_case_upper_el_dialytika:nnnn}
% \begin{macro}[EXP]{\@@_change_case_upper_el_dialytika:n}
% \begin{macro}[EXP]{\@@_change_case_upper_el_hiatus:nnnnw}
% \begin{macro}[EXP]{\@@_change_case_upper_el_hiatus:nnnnN}
% \begin{macro}[EXP]{\@@_change_case_upper_el_hiatus:nnnnn}
% \begin{macro}[EXP]
%   {
%     \@@_change_case_upper_el_ypogegrammeni:n        ,
%     \@@_change_case_upper_el-x-iota_ypogegrammeni:n
%   }
% \begin{macro}[EXP]{\@@_change_case_upper_el_stress:nn}
% \begin{macro}[EXP]{\@@_change_case_upper_el_gobble:nnnw}
% \begin{macro}[EXP]{\@@_change_case_upper_el_gobble:nnnN}
% \begin{macro}[EXP]{\@@_change_case_upper_el_gobble:nnnn}
% \begin{macro}[EXP,noTF]
%   {
%     \@@_change_case_if_greek:n                   ,
%     \@@_change_case_if_greek_spacing_diacritic:n ,
%     \@@_change_case_if_greek_accent:n            ,
%     \@@_change_case_if_greek_breathing:n         ,
%     \@@_change_case_if_greek_stress:n            ,
%     \@@_change_case_if_takes_dialytika:n         ,
%     \@@_change_case_if_takes_ypogegrammeni:n
%   }
%   For Greek uppercasing, we need to know if characters \emph{in the Greek
%   range} have accents. That means doing a \textsc{nfd} conversion first, then
%   starting a search. As described by the Unicode \textsc{cldr}, Greek accents
%   need to be found \emph{after} any U+0308 (diaeresis) and are done in two
%   groups to allow for the canonical ordering. The implementation here follows
%   the data and examples from \textsc{icu}
%   (\url{https://icu.unicode.org/design/case/greek-upper}),
%   although necessarily the implementation is somewhat different. The
%   \emph{ypogegrammeni} is filtered out here as it is not actually in the
%   Greek range, so gets lost if we leave until later. The one Greek codepoint
%   we skip is the numeral sign and question mark: the first has an awkward NFD
%   for \pdfTeX{} so is best left unchanged, and the latter has issues concerning
%   how \texttt{LGR} outputs the input and output (differently!).
%    \begin{macrocode}
\cs_new:Npn \@@_change_case_upper_el:nnnnn #1#2#3#4#5
  {
    \bool_lazy_and:nnTF
      { \@@_change_case_if_greek_p:n {#5} }
      {
        ! \bool_lazy_or_p:nn
          { \@@_codepoint_compare_p:nNn {#5} = { "0374 } }
          { \@@_codepoint_compare_p:nNn {#5} = { "037E } }
      }
      {
        \@@_change_case_if_greek_spacing_diacritic:nTF {#5}
          {
            \@@_change_case_store:n {#5}
            \@@_change_case_loop:nnnw
          }
          {
            \exp_args:Ne \@@_change_case_upper_el:nnnn
              {
                \codepoint_to_nfd:n
                  { \@@_codepoint_from_chars:Nw #5 }
              }
          }
            {#2} {#3} {#4}
      }
      {
        \@@_codepoint_compare:nNnTF {#5} = { "0345 }
          {
            \@@_change_case_store:e
              {
                \codepoint_generate:nn { "0399 }
                  { \char_value_catcode:n { "0399 } }
              }
            \@@_change_case_loop:nnnw {#2} {#3} {#4}
          }
          { \@@_change_case_codepoint:nnnnn {#1} {#2} {#3} {#4} {#5} }
      }
  }
\cs_new_eq:cN { @@_change_case_upper_el-x-iota:nnnnn }
  \@@_change_case_upper_el:nnnnn
\cs_new:Npn \@@_change_case_upper_el:nnnn #1#2#3#4
  {
    \@@_codepoint_process:nN
      { \@@_change_case_upper_el:nnnnw {#2} {#3} {#4} } #1
  }
%    \end{macrocode}
%   At this stage we have the first NFD codepoint as |#3|. What we need to know
%   is whether after that we have another character, either from the NFD or
%   directly in the input. If not, we store the changed character at this stage.
%    \begin{macrocode}
\cs_new:Npn \@@_change_case_upper_el:nnnnw #1#2#3#4#5 \q_@@_recursion_stop
  {
    \tl_if_head_is_N_type:nTF {#5}
      { \@@_change_case_upper_el:nnnnN {#4} }
      {
        \@@_change_case_store:e
          { \@@_change_case_codepoint:nn { upper } {#4} }
        \@@_change_case_loop:nnnw
      }
        {#1} {#2} {#3} #5 \q_@@_recursion_stop
  }
%    \end{macrocode}
%   Now, we check the detail of the next codepoint: again we filter out the
%   not-a-char cases, before checking if it's an dialytika, accent or diacritic.
%   (The latter do not have the same hiatus behavior as accents.) There is
%   additional work if the codepoint can take a ypogegrammeni: there, we need
%   to move any ypogegrammeni to after accents (in case the input is not
%   normalised). The ypogegrammeni itself is handled separately.
%    \begin{macrocode}
\cs_new:Npn \@@_change_case_upper_el:nnnnN #1#2#3#4#5
  {
    \token_if_cs:NTF #5
      {
        \@@_change_case_store:e
          { \@@_change_case_codepoint:nn { upper } {#1} }
        \@@_change_case_loop:nnnw {#2} {#3} {#4} #5
      }
      {
        \@@_change_case_if_takes_ypogegrammeni:nTF {#1}
          {
            \@@_change_case_upper_el_ypogegrammeni:nnnnnnw
              {#1} {#2} {#3} {#4} { } { } #5
          }
          { \@@_change_case_upper_el_aux:nnnnN {#1} {#2} {#3} {#4} #5 }
      }
  }
\cs_new:Npn \@@_change_case_upper_el_ypogegrammeni:nnnnnnw
  #1#2#3#4#5#6#7 \q_@@_recursion_stop
  {
    \tl_if_head_is_N_type:nTF {#7}
      {
        \@@_change_case_upper_el_ypogegrammeni:nnnnnnN
          {#1} {#2} {#3} {#4} {#5} {#6}
      }
      { \@@_change_case_upper_el_aux:nnnnN {#1} {#2} {#3} {#4} #5#6 }
        #7 \q_@@_recursion_stop
  }
\cs_new:Npn \@@_change_case_upper_el_ypogegrammeni:nnnnnnN #1#2#3#4#5#6#7
  {
    \token_if_cs:NTF #7
      { \@@_change_case_upper_el_aux:nnnnN {#1} {#2} {#3} {#4} #5#6 }
      {
        \@@_codepoint_process:nN
          {
            \@@_change_case_upper_el_ypogegrammeni:nnnnnnn
              {#1} {#2} {#3} {#4} {#5} {#6}
          }
      }
        #7
  }
\cs_new:Npn \@@_change_case_upper_el_ypogegrammeni:nnnnnnn #1#2#3#4#5#6#7
  {
    \@@_codepoint_compare:nNnTF {#7} = { "0345 }
      {
        \@@_change_case_upper_el_ypogegrammeni:nnnnnnw
          {#1} {#2} {#3} {#4} {#5} {#7}
      }
      {
        \bool_lazy_or:nnTF
          { \@@_change_case_if_greek_accent_p:n {#7} }
          { \@@_change_case_if_greek_breathing_p:n {#7} }
          {
            \@@_change_case_upper_el_ypogegrammeni:nnnnnnw
              {#1} {#2} {#3} {#4} {#5#7} {#6}
          }
          { \@@_change_case_upper_el_aux:nnnnN {#1} {#2} {#3} {#4} #5#6 #7 }
      }
  }
\cs_new:Npn \@@_change_case_upper_el_aux:nnnnN #1#2#3#4#5
  {
    \@@_codepoint_process:nN
      { \@@_change_case_upper_el_aux:nnnnn {#1} {#2} {#3} {#4} } #5
  }
\cs_new:Npn \@@_change_case_upper_el_aux:nnnnn #1#2#3#4#5
  {
    \@@_codepoint_compare:nNnTF {#5} = { "0308 }
      { \@@_change_case_upper_el_dialytika:nnnn {#2} {#3} {#4} {#1} }
      {
        \@@_change_case_if_greek_accent:nTF {#5}
          { \@@_change_case_upper_el_hiatus:nnnnw {#2} {#3} {#4} {#1} }
          {
            \@@_change_case_if_greek_breathing:nTF {#5}
              { \@@_change_case_upper_el:nnnn {#1} {#2} {#3} {#4} }
              {
                \@@_codepoint_compare:nNnTF {#5} = { "0345 }
                  {
                    \@@_change_case_store:e
                      { \use:c { @@_change_case_upper_ #4 _ypogegrammeni:n } {#1} }
                    \@@_change_case_loop:nnnw {#2} {#3} {#4}
                  }
                  {
                    \@@_change_case_if_greek_stress:nTF {#5}
                      {
                        \@@_change_case_store:e
                          { \@@_change_case_upper_el_stress:nn {#1} {#5} }
                        \@@_change_case_loop:nnnw {#2} {#3} {#4}

                      }
                      {
                        \@@_change_case_store:e
                          { \@@_change_case_codepoint:nn { upper } {#1} }
                        \@@_change_case_loop:nnnw {#2} {#3} {#4} #5
                      }
                  }
              }
          }
      }
  }
%    \end{macrocode}
%   We handle \emph{dialytika} in parts as it's also needed for the hiatus.
%   We know only two letters take it, so we can shortcut here on the second
%   part of the tests.
%    \begin{macrocode}
\cs_new:Npn \@@_change_case_upper_el_dialytika:nnnn #1#2#3#4
  {
    \@@_change_case_if_takes_dialytika:nTF {#4}
      { \@@_change_case_upper_el_dialytika:n {#4} }
      {
        \@@_change_case_store:e
          { \@@_change_case_codepoint:nn { upper } {#4} }
      }
    \@@_change_case_upper_el_gobble:nnnw {#1} {#2} {#3}
  }
\cs_new:Npn \@@_change_case_upper_el_dialytika:n #1
  {
    \@@_change_case_store:e
      {
        \bool_lazy_or:nnTF
          { \@@_codepoint_compare_p:nNn {#1} = { "0399 } }
          { \@@_codepoint_compare_p:nNn {#1} = { "03B9 } }
          {
            \codepoint_generate:nn { "03AA }
              { \@@_change_case_catcode:nn {#1} { "03AA } }
          }
          {
            \codepoint_generate:nn { "03AB }
              { \@@_change_case_catcode:nn {#1} { "03AB } }
          }
      }
  }
%    \end{macrocode}
%   Adding a hiatus needs some of the same ideas, but if there is not one we
%   skip this code point, hence needing a separate function.
%    \begin{macrocode}
\cs_new:Npn \@@_change_case_upper_el_hiatus:nnnnw
  #1#2#3#4#5 \q_@@_recursion_stop
  {
    \tl_if_head_is_N_type:nTF {#5}
      { \@@_change_case_upper_el_hiatus:nnnnN {#4} }
      {
        \@@_change_case_store:e
          { \@@_change_case_codepoint:nn { upper } {#4} }
        \@@_change_case_loop:nnnw
      }
        {#1} {#2} {#3} #5 \q_@@_recursion_stop
  }
\cs_new:Npn \@@_change_case_upper_el_hiatus:nnnnN #1#2#3#4#5
  {
    \token_if_cs:NTF #5
      {
        \@@_change_case_store:e
          { \@@_change_case_codepoint:nn { upper } {#1} }
        \@@_change_case_loop:nnnw {#2} {#3} {#4} #5 
      }
      {
        \@@_codepoint_process:nN
          { \@@_change_case_upper_el_hiatus:nnnnn {#1} {#2} {#3} {#4} } #5
      }
  }
\cs_new:Npn \@@_change_case_upper_el_hiatus:nnnnn #1#2#3#4#5
  {
    \@@_change_case_if_takes_dialytika:nTF {#5}
      {
        \@@_change_case_store:e
          { \@@_change_case_codepoint:nn { upper } {#1} }
        \@@_change_case_upper_el_dialytika:n {#5}
        \@@_change_case_upper_el_gobble:nnnw {#2} {#3} {#4}
      }
      { \@@_change_case_upper_el:nnnn {#1} {#2} {#3} {#4} #5 }
  }
%    \end{macrocode}
%   Handling the \emph{ypogegrammeni} output depends on the selected approach
%    \begin{macrocode}
\cs_new:Npn \@@_change_case_upper_el_ypogegrammeni:n #1
  {
    \exp_args:Ne \@@_change_case_generate:n
      {
        \int_case:nn
          { \@@_codepoint_from_chars:Nw #1 }
          {
            { "0391 } { "1FBC }
            { "03B1 } { "1FBC }
            { "0397 } { "1FCC }
            { "03B7 } { "1FCC }
            { "03A9 } { "1FFC }
            { "03C9 } { "1FFC }
          }
      }
  }
\cs_new:cpn { @@_change_case_upper_el-x-iota_ypogegrammeni:n } #1
  {
    \@@_change_case_codepoint:nn { upper } {#1}
    \codepoint_generate:nn { "0399 }
      { \char_value_catcode:n { "0399 } }
  }
%    \end{macrocode}
%   We choose to retain stress diacritics, but we also need to recombine
%   them for pdf\TeX{}. That is handled here.
%    \begin{macrocode}
\cs_new:Npn \@@_change_case_upper_el_stress:nn #1#2
  {
    \exp_args:Ne \@@_change_case_generate:n
      {
        \int_case:nn
          { \@@_codepoint_from_chars:Nw #2 }
          {
            { "0304 }
              {
                \int_case:nn { \@@_codepoint_from_chars:Nw #1 }
                  {
                    { "0391 } { "1FB9 }
                    { "03B1 } { "1FB9 }
                    { "0399 } { "1FD9 }
                    { "03B9 } { "1FD9 }
                    { "03A5 } { "1FE9 }
                    { "03C5 } { "1FE9 }
                  }
              }
            { "0306 }
              {
                \int_case:nn { \@@_codepoint_from_chars:Nw #1 }
                  {
                    { "0391 } { "1FB8 }
                    { "03B1 } { "1FB8 }
                    { "0399 } { "1FD8 }
                    { "03B9 } { "1FD8 }
                    { "03A5 } { "1FE8 }
                    { "03C5 } { "1FE8 }
                  }
              }
          }
      }
  }
%    \end{macrocode}
%   For clearing out trailing combining marks after we have dealt with
%   the first one.
%    \begin{macrocode}
\cs_new:Npn \@@_change_case_upper_el_gobble:nnnw
  #1#2#3#4 \q_@@_recursion_stop
  {
    \tl_if_head_is_N_type:nTF {#4}
      { \@@_change_case_upper_el_gobble:nnnN }
      { \@@_change_case_loop:nnnw }
        {#1} {#2} {#3} #4 \q_@@_recursion_stop
  }
\cs_new:Npn \@@_change_case_upper_el_gobble:nnnN #1#2#3#4
  {
    \token_if_cs:NTF #4
      { \@@_change_case_loop:nnnw {#1} {#2} {#3} }
      {
        \@@_codepoint_process:nN
          { \@@_change_case_upper_el_gobble:nnnn {#1} {#2} {#3} }
      }
        #4
  }
\cs_new:Npn \@@_change_case_upper_el_gobble:nnnn #1#2#3#4
  {
    \bool_lazy_or:nnTF
      { \@@_change_case_if_greek_accent_p:n {#4} }
      { \@@_change_case_if_greek_breathing_p:n {#4} }
      { \@@_change_case_upper_el_gobble:nnnw {#1} {#2} {#3} }
      { \@@_change_case_loop:nnnw {#1} {#2} {#3} #4 }
  }
%    \end{macrocode}
%   Luckily the Greek range is limited and clear.
%    \begin{macrocode}
\prg_new_conditional:Npnn \@@_change_case_if_greek:n #1 { p , TF }
  {
    \exp_args:Nf \@@_change_case_if_greek:n
      { \int_eval:n { \@@_codepoint_from_chars:Nw #1 } }
  }
\cs_new:Npn \@@_change_case_if_greek:n #1
  {
    \if_int_compare:w #1 < "0370 \exp_stop_f:
      \prg_return_false:
    \else:
      \if_int_compare:w #1 > "03FF \exp_stop_f:
        \if_int_compare:w #1 < "1F00 \exp_stop_f:
          \prg_return_false:
        \else:
          \if_int_compare:w #1 > "1FFF \exp_stop_f:
            \if_int_compare:w #1 = "2126 \exp_stop_f:
              \prg_return_true:
            \else:
              \prg_return_false:
            \fi:
          \else:
            \prg_return_true:
          \fi:
        \fi:
      \else:
        \prg_return_true:
      \fi:
    \fi:
  }
%    \end{macrocode}
%   We follow ICU in adding a few extras to the accent list here.
%    \begin{macrocode}
\prg_new_conditional:Npnn \@@_change_case_if_greek_accent:n #1 { TF , p }
  {
    \exp_args:Nf \@@_change_case_if_greek_accent:n
      { \int_eval:n { \@@_codepoint_from_chars:Nw #1 } }
  }
\cs_new:Npn \@@_change_case_if_greek_accent:n #1
  {
    \if_int_compare:w #1 = "0300 \exp_stop_f:
      \prg_return_true:
    \else:
      \if_int_compare:w #1 = "0301 \exp_stop_f:
        \prg_return_true:
      \else:
        \if_int_compare:w #1 = "0342 \exp_stop_f:
          \prg_return_true:
        \else:
          \if_int_compare:w #1 = "0302 \exp_stop_f:
            \prg_return_true:
          \else:
            \if_int_compare:w #1 = "0303 \exp_stop_f:
              \prg_return_true:
            \else:
              \if_int_compare:w #1 = "0311 \exp_stop_f:
                \prg_return_true:
              \else:
                \prg_return_false:
              \fi:
            \fi:
          \fi:
        \fi:
      \fi:
    \fi:
  }
\prg_new_conditional:Npnn \@@_change_case_if_greek_spacing_diacritic:n
  #1 { TF }
  {
    \exp_args:Nf \@@_change_case_if_greek_spacing_diacritic:n
      { \int_eval:n { \@@_codepoint_from_chars:Nw #1 } }
  }
\cs_new:Npn \@@_change_case_if_greek_spacing_diacritic:n #1
  {
    \if_int_compare:w #1 < "1FBD \exp_stop_f:
      \if_int_compare:w #1 = "037A \exp_stop_f:
        \prg_return_true:
      \else:
        \prg_return_false:
      \fi:
    \else:
      \if_int_compare:w #1 = "1FBD \exp_stop_f:
        \prg_return_true:
      \else:
        \if_int_compare:w #1 = "1FBF \exp_stop_f:
          \prg_return_true:
        \else:
          \if_int_compare:w #1 = "1FC0 \exp_stop_f:
            \prg_return_true:
          \else:
            \if_int_compare:w #1 = "1FC1 \exp_stop_f:
              \prg_return_true:
            \else:
              \if_int_compare:w #1 = "1FCD \exp_stop_f:
                \prg_return_true:
              \else:
                \if_int_compare:w #1 = "1FCE \exp_stop_f:
                  \prg_return_true:
                \else:
                  \if_int_compare:w #1 = "1FCF \exp_stop_f:
                    \prg_return_true:
                  \else:
                    \if_int_compare:w #1 = "1FDD \exp_stop_f:
                      \prg_return_true:
                    \else:
                      \if_int_compare:w #1 = "1FDE \exp_stop_f:
                        \prg_return_true:
                      \else:
                        \if_int_compare:w #1 = "1FDF \exp_stop_f:
                          \prg_return_true:
                        \else:
                          \if_int_compare:w #1 = "1FED \exp_stop_f:
                            \prg_return_true:
                          \else:
                            \if_int_compare:w #1 = "1FEE \exp_stop_f:
                              \prg_return_true:
                            \else:
                              \if_int_compare:w #1 = "1FEF \exp_stop_f:
                                \prg_return_true:
                              \else:
                                \if_int_compare:w #1 = "1FFD \exp_stop_f:
                                  \prg_return_true:
                                \else:
                                  \if_int_compare:w #1 = "1FFE \exp_stop_f:
                                    \prg_return_true:
                                  \else:
                                    \prg_return_false:
                                  \fi:
                                \fi:
                              \fi:
                            \fi:
                          \fi:
                        \fi:
                      \fi:
                    \fi:
                  \fi:
                \fi:
              \fi:
            \fi:
          \fi:
        \fi:
      \fi:
    \fi:
  }
\prg_new_conditional:Npnn \@@_change_case_if_greek_breathing:n
  #1 { TF , p }
  {
    \exp_args:Nf \@@_change_case_if_greek_breathing:n
      { \int_eval:n { \@@_codepoint_from_chars:Nw #1 } }
  }
\cs_new:Npn \@@_change_case_if_greek_breathing:n #1
  {
    \if_int_compare:w #1 = "0313 \exp_stop_f:
      \prg_return_true:
    \else:
      \if_int_compare:w #1 = "0314 \exp_stop_f:
        \prg_return_true:
      \else:
        \prg_return_false:
      \fi:
    \fi:
  }
\prg_new_conditional:Npnn \@@_change_case_if_greek_stress:n
  #1 { TF , p }
  {
    \exp_args:Nf \@@_change_case_if_greek_stress:n
      { \int_eval:n { \@@_codepoint_from_chars:Nw #1 } }
  }
\cs_new:Npn \@@_change_case_if_greek_stress:n #1
  {
    \if_int_compare:w #1 = "0304 \exp_stop_f:
      \prg_return_true:
    \else:
      \if_int_compare:w #1 = "0306 \exp_stop_f:
        \prg_return_true:
      \else:
        \prg_return_false:
      \fi:
    \fi:
  }
\prg_new_conditional:Npnn \@@_change_case_if_takes_dialytika:n #1 { TF }
  {
    \exp_args:Nf \@@_change_case_if_takes_dialytika:n
      { \int_eval:n { \@@_codepoint_from_chars:Nw #1 } }
  }
\cs_new:Npn \@@_change_case_if_takes_dialytika:n #1
  {
    \if_int_compare:w #1 = "0399 \exp_stop_f:
      \prg_return_true:
    \else:
      \if_int_compare:w #1 = "03B9 \exp_stop_f:
        \prg_return_true:
      \else:
        \if_int_compare:w #1 = "03A5 \exp_stop_f:
          \prg_return_true:
        \else:
          \if_int_compare:w #1 = "03C5 \exp_stop_f:
            \prg_return_true:
          \else:
            \prg_return_false:
          \fi:
        \fi:
      \fi:
    \fi:
  }
\prg_new_conditional:Npnn \@@_change_case_if_takes_ypogegrammeni:n #1 { TF }
  {
    \exp_args:Nf \@@_change_case_if_takes_ypogegrammeni:n
      { \int_eval:n { \@@_codepoint_from_chars:Nw #1 } }
  }
\cs_new:Npn \@@_change_case_if_takes_ypogegrammeni:n #1
  {
    \if_int_compare:w #1 = "03B1 \exp_stop_f:
      \prg_return_true:
    \else:
      \if_int_compare:w #1 = "03B7 \exp_stop_f:
        \prg_return_true:
      \else:
        \if_int_compare:w #1 = "03C9 \exp_stop_f:
          \prg_return_true:
        \else:
          \prg_return_false:
        \fi:
      \fi:
    \fi:
  }
%    \end{macrocode}
% \end{macro}
% \end{macro}
% \end{macro}
% \end{macro}
% \end{macro}
% \end{macro}
% \end{macro}
% \end{macro}
% \end{macro}
% \end{macro}
% \end{macro}
% \end{macro}
% \end{macro}
% \end{macro}
% \end{macro}
% \end{macro}
% \end{macro}
% \end{macro}
% \begin{macro}[EXP]
%   {
%     \@@_change_case_boundary_upper_el:Nnnnw,
%     \@@_change_case_boundary_upper_el-x-iota:Nnnnw
%   }
% \begin{macro}[EXP]{\@@_change_case_boundary_upper_el:nnnN}
% \begin{macro}[EXP]{\@@_change_case_boundary_upper_el:nnnn}
% \begin{macro}[EXP]{\@@_change_case_boundary_upper_el:nnnnw}
%   There is one things that need special treatment at the start of
%   words in Greek. For an isolated accent \emph{eta},
%   which is handled by seeing if we have exactly one of the affected
%   codepoints followed by a space or brace group.
%    \begin{macrocode}
\cs_new:Npn \@@_change_case_boundary_upper_el:Nnnnw
  #1#2#3#4#5 \q_@@_recursion_stop
  {
    \tl_if_head_is_N_type:nTF {#5}
      { \@@_change_case_boundary_upper_el:nnnN }
      { \@@_change_case_loop:nnnw }
        {#2} {#3} {#4} #5 \q_@@_recursion_stop
  }
\cs_new_eq:cN { @@_change_case_boundary_upper_el-x-iota:Nnnnw }
  \@@_change_case_boundary_upper_el:Nnnnw
\cs_new:Npn \@@_change_case_boundary_upper_el:nnnN #1#2#3#4
  {
    \token_if_cs:NTF #4
      { \@@_change_case_loop:nnnw {#1} {#2} {#3} }
      {
        \@@_codepoint_process:nN
          { \@@_change_case_boundary_upper_el:nnnn {#1} {#2} {#3} }
      }
        #4
  }
\cs_new:Npn \@@_change_case_boundary_upper_el:nnnn #1#2#3#4
  {
    \bool_lazy_any:nTF
      {
        { \@@_codepoint_compare_p:nNn {#4} = { "0389 } }
        { \@@_codepoint_compare_p:nNn {#4} = { "03AE } }
        { \@@_codepoint_compare_p:nNn {#4} = { "1F22 } }
        { \@@_codepoint_compare_p:nNn {#4} = { "1F2A } }
      }
      { \@@_change_case_boundary_upper_el:nnnnw {#1} {#2} {#3} {#4} }
      { \@@_change_case_breathing:nnnn {#1} {#2} {#3} {#4} }
  }
\cs_new:Npn \@@_change_case_boundary_upper_el:nnnnw
  #1#2#3#4#5 \q_@@_recursion_stop
  {
    \tl_if_head_is_N_type:nTF {#5}
      { \@@_change_case_loop:nnnw {#1} {#2} {#3} #4 }
      {
        \@@_change_case_store:e
          {
            \codepoint_generate:nn { "0389 }
              { \@@_change_case_catcode:nn {#4} { "0389 } }
          }
        \@@_change_case_loop:nnnw {#1} {#2} {#3}
      }
        #5 \q_@@_recursion_stop
  }
%    \end{macrocode}
% \end{macro}
% \end{macro}
% \end{macro}
% \end{macro}
% \begin{macro}[EXP]{\@@_change_case_breathing:nnnn}
% \begin{macro}[EXP]{\@@_change_case_breathing:nnnnn}
% \begin{macro}[EXP]{\@@_change_case_breathing:nnnnnw}
% \begin{macro}[EXP]{\@@_change_case_breathing:nnnnnnw}
% \begin{macro}[EXP]{\@@_change_case_breathing_aux:nnnnnn}
% \begin{macro}[EXP]{\@@_change_case_breathing_aux:nnnnw}
% \begin{macro}[EXP]{\@@_change_case_breathing_aux:nnnN}
% \begin{macro}[EXP]{\@@_change_case_breathing_dialytika:nnnn}
%   In Greek, breathing diacritics are normally dropped when uppercasing:
%   see the code for the general case. However, for the first character
%   of a word, if there is a breather \emph{and} the next character takes
%   a \emph{dialytika}, it needs to be added. We start by checking if
%   the current codepoint is in the Greek range, then decomposing.
%    \begin{macrocode}
\cs_new:Npn \@@_change_case_breathing:nnnn #1#2#3#4
  {
    \@@_change_case_if_greek:nTF {#4}
      {
        \exp_args:Ne \@@_change_case_breathing:nnnnn
          {
            \codepoint_to_nfd:n
              { \@@_codepoint_from_chars:Nw #4 }
          }
            {#1} {#2} {#3} {#4}
      }
      { \@@_change_case_loop:nnnw {#1} {#2} {#3} #4 }
  }
\cs_new:Npn \@@_change_case_breathing:nnnnn #1#2#3#4#5
  {
    \@@_codepoint_process:nN
      { \@@_change_case_breathing:nnnnnw {#2} {#3} {#4} {#5} }
        #1 \q_mark
  }
%    \end{macrocode}
%   Normal form decomposition will always give between one and three
%   codepoints. Luckily, the two breathing marks (\emph{psili} and
%   \emph{dasia}) will be in a predictable position: last. So we can
%   quickly establish first that there was a change on decomposition,
%   and second if the final resulting codepoint is one of the two we
%   care about.
%    \begin{macrocode}
\cs_new:Npn \@@_change_case_breathing:nnnnnw #1#2#3#4#5#6 \q_mark
  {
    \tl_if_blank:nTF {#6}
      { \@@_change_case_loop:nnnw {#1} {#2} {#3} #4 }
      {
        \@@_codepoint_process:nN
          { \@@_change_case_breathing:nnnnnnw {#1} {#2} {#3} {#4} {#5} }
            #6 \q_mark
      }
  }
\cs_new:Npn \@@_change_case_breathing:nnnnnnw #1#2#3#4#5#6#7 \q_mark
  {
    \tl_if_blank:nTF {#7}
      {
        \@@_change_case_breathing_aux:nnnnnn
          {#1} {#2} {#3} {#4} {#5} {#6}
      }
      {
        \@@_codepoint_process:nN
          { \@@_change_case_breathing:nnnnnnw {#1} {#2} {#3} {#4} {#5} }
          #7 \q_mark
      }
  }
\cs_new:Npn \@@_change_case_breathing_aux:nnnnnn #1#2#3#4#5#6
  {
    \bool_lazy_or:nnTF
      { \@@_codepoint_compare_p:nNn {#6} = { "0313 } }
      { \@@_codepoint_compare_p:nNn {#6} = { "0314 } }
      { \@@_change_case_breathing_aux:nnnnw {#1} {#2} {#3} {#5} }
      { \@@_change_case_loop:nnnw {#1} {#2} {#3} #4 }
  }
%    \end{macrocode}
%   Now the lookahead can be fired: check the next codepoint and assess
%   whether it takes a \emph{dialytika}. Drop the 
%    breathing mark or generate the \emph{dialytika}: the
%   latter is code shared with the general mechanism.
%    \begin{macrocode}
\cs_new:Npn \@@_change_case_breathing_aux:nnnnw #1#2#3#4#5
  \q_@@_recursion_stop
  {
    \@@_change_case_store:e
      { \@@_change_case_codepoint:nn { upper } {#4} }
    \tl_if_head_is_N_type:nTF {#5}
      { \@@_change_case_breathing_aux:nnnN }
      { \@@_change_case_loop:nnnw }
        {#1} {#2} {#3} #5 \q_@@_recursion_stop
  }
\cs_new:Npn \@@_change_case_breathing_aux:nnnN #1#2#3#4
  {
    \@@_codepoint_process:nN
      { \@@_change_case_breathing_dialytika:nnnn {#1} {#2} {#3} } #4
  }
\cs_new:Npn \@@_change_case_breathing_dialytika:nnnn #1#2#3#4
  {
    \@@_change_case_if_takes_dialytika:nTF {#4}
      {
        \@@_change_case_upper_el_dialytika:n {#4}
        \@@_change_case_loop:nnnw {#1} {#2} {#3}
      }
      { \@@_change_case_loop:nnnw {#1} {#2} {#3} #4 }
  }
%    \end{macrocode}
% \end{macro}
% \end{macro}
% \end{macro}
% \end{macro}
% \end{macro}
% \end{macro}
% \end{macro}
% \end{macro}
% \begin{macro}[EXP]{\@@_change_case_title_el:nnnnn}
%   Titlecasing retains accents, but to prevent the uppercasing code
%   from kicking in, there has to be an explicit function here.
%    \begin{macrocode}
\cs_new:Npn \@@_change_case_title_el:nnnnn #1#2#3#4#5
  { \@@_change_case_codepoint:nnnnn {#1} {#2} {#3} {#4} {#5} }
%    \end{macrocode}
% \end{macro}
%
% \begin{macro}[EXP]
%   {
%     \@@_change_case_upper_hy:nnnnn        ,
%     \@@_change_case_title_hy:nnnnn        ,
%     \@@_change_case_upper_hy-x-yiwn:nnnnn ,
%     \@@_change_case_title_hy-x-yiwn:nnnnn
%   }
%     See \url{https://www.unicode.org/L2/L2020/20143-armenian-ech-yiwn.pdf}.
%    \begin{macrocode}
\cs_new:Npn \@@_change_case_upper_hy:nnnnn #1#2#3#4#5
  {
    \@@_codepoint_compare:nNnTF {#5} = { "0587 }
      {
        \@@_change_case_store:e
          {
            \codepoint_generate:nn { "0535 }
              { \@@_change_case_catcode:nn {#5} { "0535 } }
            \codepoint_generate:nn { "054E }
              { \@@_change_case_catcode:nn {#5} { "054E } }
          }
        \use:c { @@_change_case_next_ #2 :nnn }
          {#2} {#3} {#4}
      }
      { \@@_change_case_codepoint:nnnnn {#1} {#2} {#3} {#4} {#5} }
  }
\cs_new:Npn \@@_change_case_title_hy:nnnnn #1#2#3#4#5
  {
    \@@_codepoint_compare:nNnTF {#5} = { "0587 }
      {
        \@@_change_case_store:e
          {
            \codepoint_generate:nn { "0535 }
              { \@@_change_case_catcode:nn {#5} { "0535 } }
            \codepoint_generate:nn { "057E }
              { \@@_change_case_catcode:nn {#5} { "057E } }
          }
        \use:c { @@_change_case_next_ #2 :nnn }
          {#2} {#3} {#4}
      }
      { \@@_change_case_codepoint:nnnnn {#1} {#2} {#3} {#4} {#5} }
  }
\cs_new:cpn { @@_change_case_upper_hy-x-yiwn:nnnnn } #1#2#3#4#5
  { \@@_change_case_codepoint:nnnnn {#1} {#2} {#3} {#4} {#5} }
\cs_new_eq:cc { @@_change_case_title_hy-x-yiwn:nnnnn }
  { @@_change_case_upper_hy-x-yiwn:nnnnn }
%    \end{macrocode}
% \end{macro}
%
% \begin{macro}[EXP]{\@@_change_case_lower_la-x-medieval:nnnnn}
% \begin{macro}[EXP]{\@@_change_case_upper_la-x-medieval:nnnnn}
%   Simply swaps of characters.
%    \begin{macrocode}
\cs_new:cpn { @@_change_case_lower_la-x-medieval:nnnnn } #1#2#3#4#5
  {
    \@@_codepoint_compare:nNnTF {#5} = { "0056 }
      {
        \@@_change_case_store:e
          { \char_generate:nn { "0075 } { \@@_char_catcode:N #5 } }
        \use:c { @@_change_case_next_ #2 :nnn }
          {#2} {#3} {#4}
      }
      { \@@_change_case_codepoint:nnnnn {#1} {#2} {#3} {#4} {#5} }
  }
\cs_new:cpn { @@_change_case_upper_la-x-medieval:nnnnn } #1#2#3#4#5
  {
    \@@_codepoint_compare:nNnTF {#5} = { "0075 }
      {
        \@@_change_case_store:e
          { \char_generate:nn { "0056 } { \@@_char_catcode:N #5 } }
        \use:c { @@_change_case_next_ #2 :nnn }
          {#2} {#3} {#4}
      }
      { \@@_change_case_codepoint:nnnnn {#1} {#2} {#3} {#4} {#5} }
  }
%    \end{macrocode}
% \end{macro}
% \end{macro}
%
% \begin{macro}[EXP]
%   {
%     \@@_change_cases_lower_lt:nnnnn      ,
%     \@@_change_cases_lower_lt_auxi:nnnnn ,
%     \@@_change_cases_lower_lt_auxii:nnnnn
%   }
% \begin{macro}[rEXP]{\@@_change_case_lower_lt:nnnw}
% \begin{macro}[rEXP]{\@@_change_case_lower_lt:nnnN}
% \begin{macro}[rEXP]{\@@_change_case_lower_lt:nnnn}
%   For  Lithuanian, the issue to be dealt with is dots over lower case
%   letters: these should be present if there is another accent. The first step
%   is a simple match attempt: look for the three uppercase accented letters
%   which should gain a dot-above char in their lowercase form.
%    \begin{macrocode}
\cs_new:Npn \@@_change_case_lower_lt:nnnnn #1#2#3#4#5
  {
    \exp_args:Ne \@@_change_case_lower_lt_auxi:nnnnn
      {
        \int_case:nn { \@@_codepoint_from_chars:Nw #5 }
          {
            { "00CC } { "0300 }
            { "00CD } { "0301 }
            { "0128 } { "0303 }
          }  
      }
        {#2} {#3} {#4} {#5}
  }
%    \end{macrocode}
%   If there was a hit, output the result with the dot-above and move on.
%   Otherwise, look for one of the three letters that can take a combining
%   accent: I, J nd I-ogonek. 
%    \begin{macrocode}
\cs_new:Npn \@@_change_case_lower_lt_auxi:nnnnn #1#2#3#4#5
  {
    \tl_if_blank:nTF {#1}
      {
        \exp_args:Ne \@@_change_case_lower_lt_auxii:nnnnn
          {
            \int_case:nn { \@@_codepoint_from_chars:Nw #5 }
              {
                { "0049 } { "0069 }
                { "004A } { "006A }
                { "012E } { "012F }
              }  
          }
            {#2} {#3} {#4} {#5}
      }
      {
        \@@_change_case_store:e
          {
            \codepoint_generate:nn { "0069 }
              { \@@_change_case_catcode:nn {#5} { "0069 } }
            \codepoint_generate:nn { "0307 }
              { \@@_change_case_catcode:nn {#5} { "0307 } }
            \codepoint_generate:nn {#1}
              { \@@_change_case_catcode:nn {#5} {#1} }
          }
        \@@_change_case_loop:nnnw {#2} {#3} {#4}
      }
  }
%    \end{macrocode}
%   Again, branch depending on a hit. If there is one, we output the character
%   then need to look for a combining accent: as usual, we need to be aware of
%   the loop situation.
%    \begin{macrocode}
\cs_new:Npn \@@_change_case_lower_lt_auxii:nnnnn #1#2#3#4#5
  {
    \tl_if_blank:nTF {#1}
      { \@@_change_case_codepoint:nnnnn {#2} {#2} {#3} {#4} {#5} }
      {
        \@@_change_case_store:e
          {
            \codepoint_generate:nn {#1}
              { \@@_change_case_catcode:nn {#5} {#1} }
          }
        \@@_change_case_lower_lt:nnnw {#2} {#3} {#4}
      }
  }
\cs_new:Npn \@@_change_case_lower_lt:nnnw #1#2#3#4 \q_@@_recursion_stop
  {
    \tl_if_head_is_N_type:nTF {#4}
      { \@@_change_case_lower_lt:nnnN }
      { \@@_change_case_loop:nnnw }
        {#1} {#2} {#3} #4 \q_@@_recursion_stop
  }
\cs_new:Npn \@@_change_case_lower_lt:nnnN #1#2#3#4
  {
    \@@_codepoint_process:nN
      { \@@_change_case_lower_lt:nnnn {#1} {#2} {#3} } #4
  }
\cs_new:Npn \@@_change_case_lower_lt:nnnn #1#2#3#4
  {
    \bool_lazy_and:nnT
      {
        \bool_lazy_or_p:nn
          { ! \tl_if_single_p:n {#4} }
          { ! \token_if_cs_p:N #4 }
      }
      {
        \bool_lazy_any_p:n
          {
            { \@@_codepoint_compare_p:nNn {#4} = { "0300 } }
            { \@@_codepoint_compare_p:nNn {#4} = { "0301 } }
            { \@@_codepoint_compare_p:nNn {#4} = { "0303 } }
          }
      }
      {
        \@@_change_case_store:e
          {
            \codepoint_generate:nn { "0307 }
              { \@@_change_case_catcode:nn {#4} { "0307 } }
          }
      }
    \@@_change_case_loop:nnnw {#1} {#2} {#3} #4
  }
%    \end{macrocode}
% \end{macro}
% \end{macro}
% \end{macro}
% \end{macro}
% \begin{macro}[EXP]
%   {
%     \@@_change_cases_upper_lt:nnnnn     ,
%     \@@_change_cases_upper_lt_aux:nnnnn
%   }
% \begin{macro}[rEXP]{\@@_change_case_upper_lt:nnnw}
% \begin{macro}[rEXP]{\@@_change_case_upper_lt:nnnN}
% \begin{macro}[rEXP]{\@@_change_case_upper_lt:nnnn}
%   The uppercasing version: first find i/j/i-ogonek, then look for the
%   combining char: drop it if present.
%    \begin{macrocode}
\cs_new:Npn \@@_change_case_upper_lt:nnnnn #1#2#3#4#5
  {
    \exp_args:Ne \@@_change_case_upper_lt_aux:nnnnn
      {
        \int_case:nn { \@@_codepoint_from_chars:Nw #5 }
          {
            { "0069 } { "0049 }
            { "006A } { "004A }
            { "012F } { "012E }
          }  
      }
      {#2} {#3} {#4} {#5}
  }
\cs_new:Npn \@@_change_case_upper_lt_aux:nnnnn #1#2#3#4#5
  {
    \tl_if_blank:nTF {#1}
      { \@@_change_case_codepoint:nnnnn { upper } {#2} {#3} {#4} {#5} }
      {
        \@@_change_case_store:e
          {
            \codepoint_generate:nn {#1}
              { \@@_change_case_catcode:nn {#5} {#1} }
          }
        \@@_change_case_upper_lt:nnnw {#2} {#3} {#4}
      }
  }
\cs_new:Npn \@@_change_case_upper_lt:nnnw #1#2#3#4 \q_@@_recursion_stop
  {
    \tl_if_head_is_N_type:nTF {#4}
      { \@@_change_case_upper_lt:nnnN }
      { \use:c { @@_change_case_next_ #1 :nnn } }
        {#1} {#2} {#3} #4 \q_@@_recursion_stop
  }
\cs_new:Npn \@@_change_case_upper_lt:nnnN #1#2#3#4
  {
    \@@_codepoint_process:nN
      { \@@_change_case_upper_lt:nnnn {#1} {#2} {#3} } #4
  }
\cs_new:Npn \@@_change_case_upper_lt:nnnn #1#2#3#4
  {
    \bool_lazy_and:nnTF
      {
        \bool_lazy_or_p:nn
          { ! \tl_if_single_p:n {#4} }
          { ! \token_if_cs_p:N #4 }
      }
      { \@@_codepoint_compare_p:nNn {#4} = { "0307 } }
      { \use:c { @@_change_case_next_ #1 :nnn } {#1} {#2} {#3} }
      { \use:c { @@_change_case_next_ #1 :nnn } {#1} {#2} {#3} #4 }
  }
%    \end{macrocode}
% \end{macro}
% \end{macro}
% \end{macro}
% \end{macro}
%
% \begin{macro}[EXP]
%   {\@@_change_case_title_nl:nnnnn, \@@_change_case_title_nl_aux:nnnnn}
% \begin{macro}[EXP]{\@@_change_case_title_nl:nnnw}
% \begin{macro}[EXP]{\@@_change_case_title_nl:nnnN}
%   For Dutch, there is a single look-ahead test for \texttt{ij} when
%   title casing. If the appropriate letters are found, produce \texttt{IJ}
%   and gobble the \texttt{j}/\texttt{J}.
%    \begin{macrocode}
\cs_new:Npn \@@_change_case_title_nl:nnnnn #1#2#3#4#5
  {
    \tl_if_single:nTF {#5}
      { \@@_change_case_title_nl_aux:nnnnn }
      { \@@_change_case_codepoint:nnnnn }
        {#1} {#2} {#3} {#4}  {#5}
  }
\cs_new:Npn \@@_change_case_title_nl_aux:nnnnn #1#2#3#4#5
  {
    \bool_lazy_or:nnTF
      { \int_compare_p:nNn {`#5} = { "0049 } }
      { \int_compare_p:nNn {`#5} = { "0069 } }
      {
        \@@_change_case_store:e
          { \char_generate:nn { "0049 } { \@@_char_catcode:N #5 } }
        \@@_change_case_title_nl:nnnw {#2} {#3} {#4}
      }
      { \@@_change_case_codepoint:nnnnn {#1} {#2} {#3} {#4} {#5} }
  }
\cs_new:Npn \@@_change_case_title_nl:nnnw #1#2#3#4 \q_@@_recursion_stop
  {
    \tl_if_head_is_N_type:nTF {#4}
      { \@@_change_case_title_nl:nnnN }
      { \use:c { @@_change_case_next_ #1 :nnn } }
        {#1} {#2} {#3} #4 \q_@@_recursion_stop
  }
\cs_new:Npn \@@_change_case_title_nl:nnnN #1#2#3#4
  {
    \bool_lazy_and:nnTF
      { ! \token_if_cs_p:N #4 }
      {
        \bool_lazy_or_p:nn
          { \int_compare_p:nNn {`#4} = { "004A } }
          { \int_compare_p:nNn {`#4} = { "006A } }
      }
      {
        \@@_change_case_store:e
          { \char_generate:nn { "004A } { \@@_char_catcode:N #4 } }
        \use:c { @@_change_case_next_ #1 :nnn } {#1} {#2} {#3}
      }
      { \use:c { @@_change_case_next_ #1 :nnn } {#1} {#2} {#3} #4 }
  }
%    \end{macrocode}
% \end{macro}
% \end{macro}
% \end{macro}
%
% \begin{macro}[EXP]{\@@_change_case_lower_tr:nnnnn}
% \begin{macro}[EXP]{\@@_change_case_lower_tr:nnnNw}
% \begin{macro}[EXP]{\@@_change_case_lower_tr:NnnnN}
% \begin{macro}[EXP]{\@@_change_case_lower_tr:Nnnnn}
%   The Turkic languages need special treatment for dotted-i and dotless-i.
%   The lower casing rule can be expressed in terms of searching first for
%   either a dotless-I or a dotted-I. In the latter case the mapping is
%   easy, but in the former there is a second stage search.
%    \begin{macrocode}
\cs_new:Npn \@@_change_case_lower_tr:nnnnn #1#2#3#4#5
  {
    \@@_codepoint_compare:nNnTF {#5} = { "0049 }
      { \@@_change_case_lower_tr:nnnNw {#1} {#3} {#4} #5 }
      {
        \@@_codepoint_compare:nNnTF {#5} = { "0130 }
          {
            \@@_change_case_store:e
              {
                \codepoint_generate:nn { "0069 }
                  { \@@_change_case_catcode:nn {#5} { "0069 } }
              }
            \@@_change_case_loop:nnnw {#1} {#3} {#4}
          }
          { \@@_change_case_codepoint:nnnnn {#1} {#2} {#3} {#4} {#5} }
      }
  }
%    \end{macrocode}
%   After a dotless-I there may be a dot-above character. If there is then
%   a dotted-i should be produced, otherwise output a dotless-i. When the
%   combination is found both the dotless-I and the dot-above char have to
%   be removed from the input.
%    \begin{macrocode}
\cs_new:Npn \@@_change_case_lower_tr:nnnNw #1#2#3#4#5 \q_@@_recursion_stop
  {
    \tl_if_head_is_N_type:nTF {#5}
      { \@@_change_case_lower_tr:NnnnN  #4 {#1} {#2} {#3} }
      {
        \@@_change_case_store:e
          {
            \codepoint_generate:nn { "0131 }
              { \@@_change_case_catcode:nn {#4} { "0131 } }
          }
        \@@_change_case_loop:nnnw {#1} {#2} {#3}
      }
        #5 \q_@@_recursion_stop
  }
\cs_new:Npn \@@_change_case_lower_tr:NnnnN #1#2#3#4#5
  {
    \@@_codepoint_process:nN
      { \@@_change_case_lower_tr:Nnnnn #1 {#2} {#3} {#4} } #5
  }
\cs_new:Npn \@@_change_case_lower_tr:Nnnnn #1#2#3#4#5
  {
    \bool_lazy_or:nnTF
      {
        \bool_lazy_and_p:nn
          { \tl_if_single_p:n {#5} }
          { \token_if_cs_p:N #5 }
      }
      { ! \@@_codepoint_compare_p:nNn {#5} = { "0307 } }
      {
        \@@_change_case_store:e 
          {
            \codepoint_generate:nn { "0131 }
              { \@@_change_case_catcode:nn {#1} { "0131 } }
          }
        \@@_change_case_loop:nnnw {#2} {#3} {#4} #5
      }
      {
        \@@_change_case_store:e
          {
            \codepoint_generate:nn { "0069 }
              { \@@_change_case_catcode:nn {#1} { "0069 } }
          }
        \@@_change_case_loop:nnnw {#2} {#3} {#4}
      }
  }
%    \end{macrocode}
% \end{macro}
% \end{macro}
% \end{macro}
% \end{macro}
% \begin{macro}[EXP]{\@@_change_case_upper_tr:nnnnn}
%   Uppercasing is easier: just one exception with no context.
%    \begin{macrocode}
\cs_new:Npn \@@_change_case_upper_tr:nnnnn #1#2#3#4#5
  {
    \@@_codepoint_compare:nNnTF {#5} = { "0069 }
      {
        \@@_change_case_store:e
          {
            \codepoint_generate:nn { "0130 }
              { \@@_change_case_catcode:nn {#5} { "0130 } }
          }
        \use:c { @@_change_case_next_ #2 :nnn } {#2} {#3} {#4}
      }
      { \@@_change_case_codepoint:nnnnn {#1} {#2} {#3} {#4} {#5} }
  }
%    \end{macrocode}
% \end{macro}
%
% \begin{macro}[EXP]
%   {\@@_change_case_lower_az:nnnnn, \@@_change_case_upper_az:nnnnn}
%   Straight copies.
%    \begin{macrocode}
\cs_new_eq:NN \@@_change_case_lower_az:nnnnn
  \@@_change_case_lower_tr:nnnnn
\cs_new_eq:NN \@@_change_case_upper_az:nnnnn
  \@@_change_case_upper_tr:nnnnn
%    \end{macrocode}
% \end{macro}
%
% The (fixed) look-up mappings for letter-like control sequences.
%    \begin{macrocode}
\group_begin:
  \cs_set_protected:Npn \@@_change_case_setup:NN #1#2
    {
      \quark_if_recursion_tail_stop:N #1
      \tl_const:cn { c_@@_lowercase_ \token_to_str:N #1 _tl }
        { #2 }
      \tl_const:cn { c_@@_uppercase_ \token_to_str:N #2 _tl }
        { #1 }
      \@@_change_case_setup:NN
    }
  \@@_change_case_setup:NN
  \AA \aa
  \AE \ae
  \DH \dh
  \DJ \dj
  \IJ \ij
  \L  \l
  \NG \ng
  \O  \o
  \OE \oe
  \SS \ss
  \TH \th
  \q_recursion_tail ?
  \q_recursion_stop
  \tl_const:cn { c_@@_uppercase_ \token_to_str:N \i _tl } { I }
  \tl_const:cn { c_@@_uppercase_ \token_to_str:N \j _tl } { J }
\group_end:
%    \end{macrocode}
%
% To deal with possible encoding-specific extensions to \tn{@uclclist},
% we check at the end of the preamble. This will therefore only apply
% to \LaTeXe{} package mode.
%    \begin{macrocode}
\tl_if_exist:NT \@expl@finalise@setup@@@@
  {
    \tl_gput_right:Nn \@expl@finalise@setup@@@@
      {
        \tl_gput_right:Nn \@kernel@after@begindocument
          {
            \group_begin:
              \cs_set_protected:Npn \@@_change_case_setup:Nn #1#2
                {
                  \quark_if_recursion_tail_stop:N #1
                  \tl_if_single_token:nT {#2}
                    {
                      \cs_if_exist:cF
                        { c_@@_uppercase_ \token_to_str:N #1 _tl }
                        {
                          \tl_const:cn
                            { c_@@_uppercase_ \token_to_str:N #1 _tl }
                            { #2 }
                        }
                      \cs_if_exist:cF
                        { c_@@_lowercase_ \token_to_str:N #2 _tl }
                        {
                          \tl_const:cn
                            { c_@@_lowercase_ \token_to_str:N #2 _tl }
                            { #1 }
                        }
                    }
                  \@@_change_case_setup:Nn
                }
              \exp_after:wN \@@_change_case_setup:Nn \@uclclist
              \q_recursion_tail ?
              \q_recursion_stop
            \group_end:
          }
      }
  }
%    \end{macrocode}
%
% A few adjustments to case mapping for combining chars: these are not needed
% for the Unicode engines
%    \begin{macrocode}
\bool_lazy_or:nnF
  { \sys_if_engine_luatex_p: }
  { \sys_if_engine_xetex_p: }
  {
    \text_declare_uppercase_mapping:nn { "01F0 } { \v { J } }
  }
%    \end{macrocode}
%
%    \begin{macrocode}
%</package>
%    \end{macrocode}
%
% \end{implementation}
%
% \PrintIndex
